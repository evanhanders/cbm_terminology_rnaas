\section{Conclusion}
\label{sec:conclusions}
In conclusion, convective boundary mixing (CBM) is a conglomeration of a few distinct dynamical processes.
These processes include convective overshoot, entrainment, and penetrative convection.
A thorough understanding and parameterization of each of these processes can reduce discrepancies between models and observations.

We have not made any distinction between radiative zones which are \emph{thermally} stable and those which are \emph{compositionally} stable.
Composition gradients are subject to the processes discussed in this work in the same way as thermal gradients.
The only distinction is that the radiative flux, and divergences in it, act to restore the thermal gradients to their radiative values, while composition gradients have no restoring flux.

Modeling of convective boundaries has plagued stellar structure modelers for many years \citep{mesa1, mesa4, mesa5}.
Unfortunately, throughout the stellar structure literature, ``convective overshoot'', ``convective penetration'', and ``convective boundary mixing'' are often used interchangeably, which increases confusion regarding this tricky topic.
Coming to an agreement as a community about the terminology of convective boundary mixing and the processes that terminology refers to will help us pinpoint areas where models behave poorly and design experiments to improve those models.


