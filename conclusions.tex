\section{Conclusion}
\label{sec:conclusions}
Convective boundary mixing is a conglomeration of a few distinct dynamical processes.
These processes include convective overshoot, entrainment, and penetrative convection.
A thorough understanding and parameterization of each of these processes can reduce discrepancies between models and observations.

We have not made any distinction between radiative zones which are \emph{thermally} stable and those which are \emph{compositionally} stable.
Composition gradients are subject to the processes discussed in this work in the same way as thermal gradients.
The only distinction is that the radiative flux acts to restore the thermal gradients to their radiative values, while composition gradients have no restoring flux.

Stellar structure modelers have sought improved prescriptions for convective boundary mixing for many years \citep{mesa1, mesa4, mesa5}.
Unfortunately, throughout the stellar structure literature, ``convective overshoot'', ``convective penetration'', and ``convective boundary mixing'' are often used interchangeably, which increases confusion regarding this complex topic.
By reaching community agreement on both the \emph{terminology} of convective boundary mixing and the \emph{processes} that terminology refers to, we can better identify where models behave poorly and design experiments to improve them.


