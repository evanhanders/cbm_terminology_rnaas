% Preamble
\documentclass[twocolumn, linenumbers, twocolappendix]{aastex631}
%\documentclass[twocolumn]{aastex631}
\usepackage{natbib}
\usepackage{latexsym}
\usepackage{graphicx}
\usepackage{epsfig}
\usepackage{amssymb}
\usepackage{amsmath}
\usepackage{epstopdf}
\usepackage{hyperref}
\usepackage{xcolor}

%%%% Custom commands
\newcommand{\yL}{\ensuremath{\mathcal{Y}_{\rm{L}}}}
\newcommand{\yS}{\ensuremath{\mathcal{Y}_{\rm{S}}}}
\newcommand{\justgrad}{\ensuremath{\nabla}}
\newcommand{\gradrad}{\ensuremath{\nabla_{\rm{rad}}}}
\newcommand{\gradad}{\ensuremath{\nabla_{\rm{ad}}}}
\newcommand{\gradC}{\ensuremath{\nabla_{\mathrm{C}}}}
\newcommand{\gradmu}{\ensuremath{\nabla_{\mu}}}
\newcommand{\gradL}{\ensuremath{\nabla_{\mathrm{L}}}}
\newcommand{\gradT}{\ensuremath{\nabla_{\mathrm{T}}}}
\newcommand{\Ro}{\ensuremath{\mathrm{R}_{0}}}
\newcommand{\delp}{\ensuremath{\delta_{\rm{p}}}}
\newcommand{\Fbot}{\ensuremath{F_{\rm{bot}}}}
\newcommand{\Ftot}{\ensuremath{F_{\rm{tot}}}}
\newcommand{\Frad}{\ensuremath{F_{\rm{rad}}}}
\newcommand{\Fconv}{\ensuremath{F_{\rm{conv}}}}
\newcommand{\Fcz}{\ensuremath{F_{\rm{cz}}}}
\newcommand{\mP}{\ensuremath{\mathcal{P}}}
\newcommand{\mD}{\ensuremath{\mathcal{D}}}
\newcommand{\dP}{\ensuremath{\delta_{\rm{p}}}}
\newcommand{\Lcz}{\ensuremath{L_{\rm{CZ}}}}
\newcommand{\mR}{\ensuremath{\mathcal{R}}}
\newcommand{\mS}{\ensuremath{\mathcal{S}}}
\newcommand\Pran{\ensuremath{\mathrm{Pr}}}
\newcommand{\brunt}{{Brunt-V\"{a}is\"{a}l\"{a}}}

\newcommand{\angles}[1]{\langle #1 \rangle}
\newcommand{\pd}[1]{\partial_{#1}}
\renewcommand{\vec}[1]{\boldsymbol{#1}}
\newcommand{\M}[1]{\mathbf{#1}}
\renewcommand{\dot}{\vec{\cdot}}
\renewcommand{\bar}[1]{\overline{#1}}
\newcommand{\grad}{\vec{\nabla}}
\newcommand{\cross}{\vec{\times}}
\newcommand{\laplacian}{\nabla^2}

\newcommand{\editone}[1]{\textcolor{orange}{#1}}

\newcommand{\todo}[1]{ {\color{blue} \noindent\footnotesize \\\textsf{}TODO:} \textsf{#1}\\\noindent }}


%%%% Journal preamble
\received{}
\revised{}
\accepted{}
\published{}
\submitjournal{RNAAS}

\shorttitle{CBM Terminology}
\shortauthors{Anders et al}


\begin{document}

%%%% Title and Abstract
\title{A Summary of Processes in Convective Boundary Mixing}
\author[0000-0002-3433-4733]{Evan H. Anders}
\affiliation{CIERA, Northwestern University, Evanston IL 60201, USA}
\affiliation{Kavli Institute for Theoretical Physics, University of California, Santa Barbara, CA 93106, USA}
\author[0000-0001-5048-9973]{Adam S. Jermyn}
\affiliation{Center for Computational Astrophysics, Flatiron Institute, New York, NY 10010, USA}
\affiliation{Kavli Institute for Theoretical Physics, University of California, Santa Barbara, CA 93106, USA}
\author[0000-0002-7635-9728]{Daniel Lecoanet}
\affiliation{CIERA, Northwestern University, Evanston IL 60201, USA}
\affiliation{Department of Engineering Sciences and Applied Mathematics, Northwestern University, Evanston IL 60208, USA}
\affiliation{Kavli Institute for Theoretical Physics, University of California, Santa Barbara, CA 93106, USA}
\author[0000-0003-2124-9764]{J. R. Fuentes}
\affiliation{Department of Physics and McGill Space Institute, McGill University, 3600 rue University, Montreal, QC H3A 2T8, Canada}
\author{others}

\correspondingauthor{Evan H. Anders}
\email{evan.anders@northwestern.edu}

\begin{abstract}
    Convection zones have motions which extend beyond the nominal boundary of the convection zone.
    The mechanisms which lead to these motions are collectively known as ``convective boundary mixing'' (CBM).
    However, the terminology of this field is muddled in stellar astrophysics; terms like convective ``overshoot'' and ``penetration'' are often used interchangeably even though they refer to specific processes.
    Here we briefly recall the fluid dynamical processes of convective overshoot, entrainment, and convective penetration.
\end{abstract}
\keywords{Stellar convection zones (301), Stellar physics (1621); Stellar evolutionary models (2046)}


%%%% Body of paper
\section{Introduction}
\label{sec:introduction}

\begin{enumerate}
\item Observations tell us we don't understand the mixing at convective boundaries.
\item In order to resolve this problem, we need to build a community understanding of CBM processes.
\item The literature of stellar astrophysics often uses the terms ``convective overshoot'' and ``convective penetration'' interchangeably.
However, these terms refer to separate and distinct fluid dynamical processes.
\item We will briefly describe three fluid dynamical processes which may be important at the boundaries of convective regions: convective overshoot, entrainment, and convective penetration.
\end{enumerate}



%\begin{figure*}[t!]
%\centering
%\includegraphics[width=\textwidth]{dynamics_figure.pdf}
%\caption{
%\label{fig:dynamics}
%}
%\end{figure*}

\section{CBM Processes}
\label{sec:processes}

Observational inferences make it clear that 1D stellar models fail to produce realistic convective boundary mixing \citep[CBM,][]{pinsonneault_1997, claret_torres_2018, pedersen_etal_2021}.
In this section, we will briefly describe three distinct CBM processes: convective overshoot, entrainment, and penetrative convection.
In the following discussion of each CBM process, ``convective boundary'' refers to the location coinciding with the sign change of either the Schwarzschild or Ledoux discriminant.

\subsection{Convective Overshoot}
Convective overshoot occurs because the convective boundary is not the location where convective velocities are zero, but rather the location where the \emph{buoyant acceleration} of the fluid is zero.
The process of convective overshoot is shown in the upper left panel of Fig.~\ref{fig:schema}, where motions from the white convection zone (CZ) overshoot into the purple stable radiative zone (RZ).
Flows buoyantly decelerate beyond the convective boundary, so there is an extended overshoot zone (OZ) with nonzero convective velocities.

A simple $\Delta x = u \Delta t$ argument provides an estimate for how far convective motions overshoot.
Here $\Delta x$ is the overshoot distance, $u$ is the convective velocity, and $\Delta t \approx N^{-1}$ where $N$ is the \brunt$\,$frequency in the stable region.
In stellar environments, this estimate generally retrieves $\Delta x \ll H_P$, where $H_P$ is the pressure scale height.
We note that there is disagreement regarding precisely how to calculate $\Delta x$, but this estimate provides the proper flavor.

The exponential overshoot parameterization \citep[per e.g.,][]{herwig_2000} which is frequently implemented in 1D models describes this process fairly well, but 1D models generally use a much larger $\Delta x/H_P \sim \mathcal{O}(0.1)$ than hydrodynamical simulations suggest should happen when low Mach number flows encounter very stable interfaces.
Such simulations have been discussed and contextualized in e.g., \citet{korre_etal_2019}.

\begin{figure*}[t]
\centering
\includegraphics[width=\textwidth]{processes_and_structure_figure.pdf}
\caption{
    The three processes discussed in Sec.~\ref{sec:processes} are schematically demonstrated in the top row.
    White fluid has the properties of the well-mixed CZ, while purple fluid is the stable RZ.
    (Left) Convective overshoot occurs when a fluid parcel from the CZ crosses into the RZ; due to a strong positive entropy gradient in the RZ, the parcel is accelerated back into the CZ.
    (Middle) Motions generated by overshooting fluid parcels drag fluid from the RZ into the CZ in a process called entrainment.
    (Right) If a region of fluid beyond the convective boundary becomes well-mixed, it is a PZ; a divergence of the radiative flux acts as an internal cooling term, changing the buoyant signature of both upflows and downflows.
    In the bottom panel, we show the structure of a statistically-stationary convective boundary.
    The CZ sits below a well-mixed PZ.
    Above the PZ, the fluid is stable but there is a small OZ where convective overshoot occurs.
    Above this region is the stable RZ where there are internal gravity waves excited by the convection.
\label{fig:schema}
}
\end{figure*}





\subsection{Entrainment}
The process of entrainment is shown in the upper middle panel of Fig.~\ref{fig:schema}.
Return flows from overshooting convection carry fluid with the chemical and thermodynamic signature of the RZ.
This material then rapidly turbulently mixes in the convection zone.
As a result, convective motions which overshoot and entrain materials can gradually move convective boundaries.
Since entrainment is linked to convective overshooting, the overshoot distance $\Delta x$ directly relates to the \emph{rate} of entrainment \citep[which can be inferred from frequently-plotted entrainment rate laws;][]{meakin_arnett_2007}.

Entrainment has been modeled in 1D stellar evolution software instruments by \citet{staritsin_2013} and \citet{scott_etal_2021}, but their implementations differ from one another and entrainment is not standard in any instrument.
Hydrodynamical simulations of entrainment have been discussed and contextualized in e.g., \citet{fuentes_cumming_2020}.


\subsection{Penetrative Convection}
The process of penetrative convection is shown in the upper right panel of Fig.~\ref{fig:schema}.
Through continual overshoot and entrainment, convection creates well-mixed regions beyond the convective boundary.
These well-mixed extensions of the CZ are known as penetration zones (PZs).
Since PZs grow gradually over many dynamical times, they can extend beyond the overshoot distance $\Delta x$.

Penetrative convection most closely resembles ``step overshoot'' employed in 1D models, but penetrative convection mixes both entropy and composition.
Hydrodynamical simulations of penetrative convection have been discussed and contextualized in e.g., \citet{anders_etal_2021}.


\section{Conclusion}
\label{sec:conclusions}
In conclusion, convective boundary mixing (CBM) is a conglomeration of a few distinct dynamical processes.
These processes include mechanical overshoot, entrainment, and penetrative convection.
A thorough understanding and parameterization of each of these processes can reduce discrepancies between models and observations.

Modeling of convective boundaries has plagued stellar structure modelers for many years \todo{cite mesa}
Unfortunately, throughout the stellar structure literature, ``convective overshoot,'' ``convective penetration,'' and ``convective boundary mixing'' are often used interchangeably, which increases confusion regarding this tricky topic.
Coming to an agreement as a community about the terminology of convective boundary mixing and the processes that terminology refers to will help us pinpoint areas where models behave poorly and design experiments to improve those models.


\begin{acknowledgments}
EHA is funded as a CIERA Postdoctoral fellow and would like to thank CIERA and Northwestern University. 
This research was supported in part by the National Science Foundation under Grant No. PHY-1748958, and we acknowledge the hospitality of KITP during the Probes of Transport in Stars Program.
The Flatiron Institute is supported by the Simons Foundation.
\end{acknowledgments}


\bibliographystyle{aasjournal}
\bibliography{biblio}
\end{document}
