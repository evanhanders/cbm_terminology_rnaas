\section{CBM Processes}
\label{sec:processes}

We will now describe each CBM process in turn.
In the following discussion, when we say ``convective boundary,'' we are referring to the location of the convective boundary determined by the sign change of either the Schwarzschild or Ledoux discriminant.

\subsection{Mechanical Overshoot}
The process of mechanical overshoot (or convective overshoot) is shown in the upper left panel of Fig.~\ref{fig:schema}.
Mechanical overshoot occurs because the convective boundary is not the location where convective velocities are zero, but rather the location where the \emph{buoyant acceleration} of the fluid is zero.
Flows buoyantly decelerate beyond the convective boundary, so there is an extended region with nonzero convective velocities.

A simple $\Delta x = u \Delta t$ argument provides an estimate for how deeply convective motions overshoot.
Here $\Delta x$ is the overshoot depth, $u$ is the convective velocity, and $\Delta t \approx N^{-1}$ where $N$ is the \brunt$\,$frequency in the stable region.
In stellar environments, this estimate generally retrieves $\Delta x \ll H_P$, where $H_P$ is the pressure scale height.

The exponential overshoot parameterization \citep[per e.g.,][]{herwig_2000} which is implemented in many 1D stellar evolution models describes this process fairly well, but 1D modelers generally use $\Delta x/H_P \sim \mathcal{O}(0.1)$, much larger than our earlier estimate.
Hydrodynamical simulations of overshoot have been discussed and contextualized in e.g., \citet{korre_etal_2019}.


\subsection{Entrainment}
The process of entrainment is shown in the upper middle panel of Fig.~\ref{fig:schema}.
Return flows from overshooting convection carry fluid with the chemical and thermodynamic signature of the RZ.
This material then rapidly turbulently mixes in the convection zone.
As a result, convective motions which overshoot and entrain materials can cause convective boundaries to gradually advance.
As entrainment is linked to the overshooting process described above, the overshoot length scale $\Delta x$ is directly related to how rapidly a convective boundary can advance by entrainment.

Entrainment has been modeled in 1D stellar evolution codes by \citet{staritsin_2013} and \citet{scott_etal_2021}, but their implementations differ from one another and entrainment is not standard in any code.
Hydrodynamical simulations of entrainment have been discussed and contextualized in e.g., \citet{meakin_arnett_2007} \& \citet{fuentes_cumming_2020}.


\subsection{Penetrative Convection}
The process of penetrative convection is shown in the upper right panel of Fig.~\ref{fig:schema}.
Through continual overshoot and entrainment, well-mixed convective regions can extend beyond the convective boundary.
In these adiabatic extensions to the convection zone, weak buoyancy forces decelerate convective flows over appreciable length scales.
Over time, this process creates an extended, nearly-adiabatic ``penetrative zone'' beyond the convective boundary.
Since penetrative zones are established over many dynamical times, they can have length scales much larger than the previously mentioned overshoot length scale $\Delta x$.

Penetrative convection most closely resembles ``step overshoot'' employed in 1D stellar evolution models, but penetrative convection mixes both entropy and composition.
Hydrodynamical simulations of penetrative convection have been discussed and contextualized in e.g., \citet{anders_etal_2021}, following on the theory laid down in \citet{roxburgh1978, roxburgh1989} and \citet{zahn1991}.
