\section{CBM Processes}
\label{sec:processes}

In the following discussion, we will assume that the edge of the convection zone is the place where a stability criterion (Ledoux or Schwarzschild) crosses through zero.

\subsection{Convective Overshoot}
Convective overshoot is a phenomenon that occurs because the boundary of a convection zone, as determined by a discriminant's root, defines the location where the acceleration due to buoyancy changes sign.
However, it does not correspond to the place where the convective velocity is zero.
Flows buoyantly decelerate beyond the nominal convective boundary, leading to an extended region where velocities are appreciably.
This extended region is a convective overshoot.

A simple estimate of the size of an overshoot region can come from a simple $\Delta x = u \Delta t$ argument.
Here $\Delta x$ is the size of the OZ, $u$ is the convective velocity, and $\Delta t \sim N^{-1}$ is the inverse of the \brunt$\,$frequency in the RZ.
Overshoot is a process which occurs on the dynamical timescale and is ever-present, but is generally small in stellar environments.

This process is generally the process people implement in stellar structure codes \citep[per e.g.,][]{herwig_2000}, but generally occurs on much smaller length scales than is used frequently.
Convective overshoot was studied thoroughly and discussed in \citet{korre_etal_2019}


\subsection{Entrainment}
Return flows from overshooting convection carry material with the chemical and entropic composition of the RZ.
This material is quickly mixed by turbulence in the convection zone.
As a result, convective motions which overshoot and entrain materials can cause convective boundaries to gradually advance.

Entrainment is a process which occurs over many dynamical times, and is generally much faster than nuclear timescales.
Entrainment has been modeled in stellar structure codes by a few authors \citep{staritsin_2013, scott_etal_2021}, but is not standard in any codes.
Entrainment was studied and discussed in its historical context in \citet{fuentes_cumming_2020}.


\subsection{Convective penetration}
In some cases, convection zones can grow by entrainment beyond the place where $\gradrad = \gradad$.
This can create an extended, well-mixed nearly-adiabatic ``penetration zone'' beyond the convective boundary.
The process of convective penetration is therefore a process by which convection zones can potentially mix far beyond the convective boundary.

Convective penetration most closely resembles the ``step overshoot'' often employed in stellar structure models, but it mixes entropy as well as composition.
\citet{anders_etal_2021} discuss the theory of convective penetration, includings its roots in the work of \citet{roxburgh1978, roxburgh1989} and \citet{zahn1991}.


\subsection{Some points of confusion}
We note that the expansion of convective boundaries by entrainment is often referred to as conective penetration.
However, the growth of convection zones by entrainment generally involves movement of the convective boundary criterion.
These are distinct proceses, although entrainment can also lead to convective penetration.
