\section{CBM Processes}
\label{sec:processes}

In the following discussion of each CBM process, ``convective boundary'' refers to the location coinciding with the sign change of either the Schwarzschild or Ledoux discriminant.

\subsection{Mechanical Overshoot}
The process of mechanical overshoot (or convective overshoot) is shown in the upper left panel of Fig.~\ref{fig:schema}.
Mechanical overshoot occurs because the convective boundary is not the location where convective velocities are zero, but rather the location where the \emph{buoyant acceleration} of the fluid is zero.
Flows buoyantly decelerate beyond the convective boundary, so there is an extended overshoot zone (OZ) with nonzero convective velocities.

A simple $\Delta x = u \Delta t$ argument provides an estimate for how far convective motions overshoot.
Here $\Delta x$ is the overshoot distance, $u$ is the convective velocity, and $\Delta t \approx N^{-1}$ where $N$ is the \brunt$\,$frequency in the stable region.
In stellar environments, this estimate generally retrieves $\Delta x \ll H_P$, where $H_P$ is the pressure scale height.

The exponential overshoot parameterization \citep[per e.g.,][]{herwig_2000} which is frequently implemented in 1D models describes this process fairly well, but 1D models generally use a much larger $\Delta x/H_P \sim \mathcal{O}(0.1)$ than hydrodynamical simulations suggest.
Such simulations have been discussed and contextualized in e.g., \citet{korre_etal_2019}.


\subsection{Entrainment}
The process of entrainment is shown in the upper middle panel of Fig.~\ref{fig:schema}.
Return flows from overshooting convection carry fluid with the chemical and thermodynamic signature of the radiative zone (RZ).
This material then rapidly turbulently mixes in the convection zone.
As a result, convective motions which overshoot and entrain materials can gradually move convective boundaries.
Since entrainment is linked to mechanical overshooting, the overshoot distance $\Delta x$ directly relates to the \emph{rate} of entrainment.

Entrainment has been modeled in 1D stellar evolution software instruments by \citet{staritsin_2013} and \citet{scott_etal_2021}, but their implementations differ from one another and entrainment is not standard in any instrument.
Hydrodynamical simulations of entrainment have been discussed and contextualized in e.g., \citet{fuentes_cumming_2020}.


\subsection{Penetrative Convection}
The process of penetrative convection is shown in the upper right panel of Fig.~\ref{fig:schema}.
Through continual overshoot and entrainment, convection can create well-mixed regions which extend beyond the convective boundary.
In these adiabatic extensions to the convection zone, known as penetration zones (PZs), weak buoyancy forces act to decelerate convective flows.
Since PZs can grow gradually over many dynamical times, they can be much larger than the overshoot distance $\Delta x$.

Penetrative convection most closely resembles ``step overshoot'' employed in 1D models, but penetrative convection mixes both entropy and composition.
Hydrodynamical simulations of penetrative convection have been discussed and contextualized in e.g., \citet{anders_etal_2021}.
