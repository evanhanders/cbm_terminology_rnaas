\section{Convective Boundary Structure}
\label{sec:structure}
The structure of a convective boundary in its stationary state is shown in the bottom panel of Fig.~\ref{fig:schema}.
The convective boundary as determined by the Schwarzschild or Ledoux criterion is denoted by a black horizontal line.
Below that sits the convection zone, which is well-mixed as denoted by its white background, and where upflows are hot and downflows are cold.
Above the convection zone lies a well-mixed penetrative zone, where upflows are cold and downflows are hot (showing that flows are buoyantly decelerating but still have appreciable velocity).
Above the penetrative zone sits a thin, stably-stratified overshoot zone where convective motions are rapidly buoyantly restored and decelerated by a strong positive entropy gradient.
Above the overshoot zone lies a radiative zone, where waves have been excited by the convection.
Note that there is no ``entrainment zone.''
The entrainment which establishes the penetrative zone is a transient process, and while there may be some entrainment of material from the overshoot zone into the penetrative zone, that process does not have a separate region.

