\section{Convective Boundary Structure}
\label{sec:structure}
The structure of a convective boundary in steady state is shown in the bottom panel of Fig.~\ref{fig:schema}.
The convective boundary as determined by the Schwarzschild or Ledoux criterion is denoted by a black horizontal line, and the CZ is below that line.
In the CZ, upflows are hot and downflows are cold.
A well-mixed PZ is above the CZ.
In the PZ, the radiative luminosity exceeds the system luminosity, so the convective luminosity is negative, resulting in cold upflows and warm downflows.
Both entropy and composition are mixed throughout the CZ and PZ.
The upper boundary of the PZ is marked by a purple line.
Above the PZ, there is a stably-stratified OZ where composition is mixed and a strong positive entropy gradient rapidly decelerates convective motions.
A thin grey line denotes the top of the OZ.
Above the OZ is an RZ filled with internal gravity waves.
In a statistically-stationary state, entrainment is negligible.
Entrainment is the process that reconfigures a convective boundary to create the picture in Fig.~\ref{fig:schema}, but since it is a transient process there is no ``entrainment zone.''
