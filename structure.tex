\section{Convective Boundary Structure}
\label{sec:structure}
The structure of a convective boundary in its stationary state is shown in the bottom panel of Fig.~\ref{fig:schema}.
The convective boundary as determined by the Schwarzschild or Ledoux criterion is denoted by a black horizontal line, and the convective zone (CZ) is below that line.
In the CZ, upflows are hot and downflows are cold.
A well-mixed PZ is above the CZ; there, upflows are cold and downflows are hot, so flows buoyantly decelerate.
The boundary between the PZ and the stably-stratified OZ above it is marked by a purple line.
In the OZ, convective motions are rapidly buoyantly decelerated by a strong positive entropy gradient.
The top of the OZ is marked by a thin grey line, and an RZ filled with internal gravity waves lies above it.
Note that there is no ``entrainment zone;'' the entrainment which establishes the penetrative zone is a transient process, but there may still be entrainment of material from the OZ into the PZ.

